\section{Preface}
\begin{frame}
  \slidetitle

  Required skills:
  \begin{itemize}
    \item basic command line knowledge
    \item use a text editor
    \item type what is written on the projector :)
  \end{itemize}
  \vspace{1em}
  This workshop takes about 4 hours.
\end{frame}

\subsection{History}
\begin{frame}
  \subslidetitle

  \textbf{Version Control Systems (VCS)}
  \pause
  \\
  \begin{tabular}{lp{5cm}r}
    \textbf{1982} & Revision Control System (RCS) & GNU GPL \\
    \pause
    \textbf{1990} & Concurrent Versions System (CVS) & GNU GPL \\
    \pause
    \textbf{1992} & Rational ClearCase & proprietary \\
    \pause
    \textbf{1995} & Perforce  & proprietary \\
    \pause
    \textbf{2000} & Apache Subversion (SVN)  & Apache \\
    \pause
  \end{tabular}
\end{frame}

\subsection{History}
\begin{frame}
  \subslidetitle
  \textbf{Distributed Version Control Systems (DVCS)}
  \pause
  \\
  \begin{tabular}{lp{5cm}r}
    \textbf{2000} & BitKeeper  & proprietary \\
    \pause
    \textbf{2001} & GNU arch   & GNU GPL \\
    \pause
    \textbf{2003} & Monotone   & GNU GPL \\
    \pause
    \textbf{2005} & git & GNU GPL \\
    \pause
    \textbf{2005} & GNU Bazaar & GNU GPL \\
    \pause
    \textbf{2005} & Mercurial  & GNU GPL \\
    \pause
  \end{tabular}

  Linus Torvalds created git in order to replace BitKeeper which changed the license in 2005.
\end{frame}

\subsection{git}
\begin{frame}
  \subslidetitle
   The term 'git' is British slang for:
  \begin{itemize}
    \item pig headed
    \item think they are always correct
    \item argumentative
  \end{itemize}

  \epigraph{``I'm an egotistical bastard, and I name all my projects after myself. First Linux, now git.''}
       {--- Linus Torvalds, 2007-06-14}
    \epigraph{``Because my hatred of CVS has meant that I see Subversion as being the most pointless project ever started, because the whole slogan for the Subversion for a while was 'CVS done right' or something like that. And if you start with that kind of slogan, there is nowhere you can go. It's like, there is no way to do CVS right.''}
      {--- Linus Torvalds, Google Tech Talk 2007}
\end{frame}

\subsection{Design goals for git}
\begin{frame}
  \subslidetitle
  \begin{itemize}
    \item take Concurrent Versions System (CVS) as an example of what not to do; if in doubt, make the exact opposite decision
    \pause
    \item support a distributed, BitKeeper-like workflow
    \pause
    \item very strong safeguards against corruption, either accidental or malicious
    \pause
  \end{itemize}

  \vspace{2em}
  \begin{tabular}{ll}
    Started: & 2005-04-03 \\
    \pause
    Announced: &2005-04-06 \\
    \pause
    Self Hosting: & 2005-04-07 (4d!) \\
    \pause
    First Multi Branch Merge: & 2005-04-18 (15d!)
  \end{tabular}

\end{frame}

%\subsection{Course goal}
%\frame{
  %\subslidetitle

  %This course aims to provide knowledge about:
  %\begin{itemize}
    %\item Setup git (gitconfig, alias)
    %\item Clone repository (clone)
    %\item Create local changes (status, add, commit, reset, diff, log, reflog)
    %\item Work with different branches (branch, checkout, stash, diff, rebase, merge, cherry-pick)
    %\item Work with remotes (remote, fetch, pull, push, upstream, blame)
    %\item Master merge conflict (mergetool, 3 way merge)
    %\item format-patch
    %\item Find the bug with git bisect
    %\item pull requests
    %\item Advanced git: porcelain and plumbing
  %\end{itemize}
  %\vspace{1em}
%}

%\subsection{Git workflow}
%\frame{
  %\subslidetitle

  %\begin{itemize}
    %\item Fetch or clone a repository
    %\item Modify the local files
    %\item Stage the files
    %\item Commit the changes locally
    %\item Push changes to remote repository
  %\end{itemize}
  %\vspace{1em}
%}
