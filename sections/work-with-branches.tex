\section{Working with branches}
\begin{frame}[fragile]
    \slidetitle
In this section we will learn to use git branches.
\end{frame}

\subsection{The master branch}
\begin{frame}[fragile]
    \subslidetitle
\end{frame}

\subsection{Creating branches}
\begin{frame}[fragile]
    \subslidetitle

To create a new \textbf{foo} branch, append the branch name to the \cmd{git branch} command.
\begin{lstlisting}
(*\textcolor[HTML]{18B2B2}{(master)}*) $ (*\textcolor[HTML]{0000AA}{git branch foo}*)
\end{lstlisting}

The \cmd{git branch} command lists you all your local branches
\begin{lstlisting}
(*\textcolor[HTML]{18B2B2}{(master)}*) $  (*\textcolor[HTML]{0000AA}{git branch}*)
  foo
* (*\textcolor[HTML]{00AA00}{master}*)
\end{lstlisting}

Note: the * character indicate which branch we are currently working on.
\end{frame}

\subsection{Switching to branch}
\begin{frame}[fragile]
    \subslidetitle
The \cmd{git checkout} command, is used to change the working branch.
\begin{lstlisting}
(*\textcolor[HTML]{18B2B2}{(master)}*) $ (*\textcolor[HTML]{0000AA}{git checkout foo}*)
Switched to branch 'foo'
(*\textcolor[HTML]{18B2B2}{(foo)}*) $ (*\textcolor[HTML]{0000AA}{git branch}*)
* (*\textcolor[HTML]{00AA00}{foo}*)
  master
\end{lstlisting}

The \cmd{git checkout} command with \textbf{-b} option creates a new branch and automatically switch to it.
\begin{lstlisting}
(*\textcolor[HTML]{18B2B2}{(foo)}*) $ (*\textcolor[HTML]{0000AA}{git checkout -b bar}*)
Switched to a new branch 'bar'
(*\textcolor[HTML]{18B2B2}{(bar)}*) $ (*\textcolor[HTML]{0000AA}{git branch}*)
* (*\textcolor[HTML]{00AA00}{bar}*)
  foo
  master
\end{lstlisting}
\end{frame}

\subsection{Deleting a branch}
\begin{frame}[fragile]
    \subslidetitle
To delete an existing branch use the -d or -D (force) flag:
\begin{lstlisting}
(*\textcolor[HTML]{18B2B2}{(bar)}*) $ (*\textcolor[HTML]{0000AA}{git checkout master}*)
(*\textcolor[HTML]{18B2B2}{(master)}*) $ (*\textcolor[HTML]{0000AA}{git branch foo bar -d}*)
Deleted branch foo (was c974445).
Deleted branch bar (was 9adadac).
(*\textcolor[HTML]{18B2B2}{(master)}*) $ (*\textcolor[HTML]{0000AA}{git branch}*)
* (*\textcolor[HTML]{00AA00}{master}*)
\end{lstlisting}

Note: you cannot delete the branch you are currently working on.

\end{frame}

\subsection{git stash}
\begin{frame}[fragile]
    \subslidetitle
% clean, pop, 
\end{frame}

\subsection{git diff}
\begin{frame}[fragile]
    \subslidetitle
\end{frame}

\subsection{git rebase}
\begin{frame}[fragile]
    \subslidetitle
% -i
\end{frame}

\subsection{git merge}
\begin{frame}[fragile]
    \subslidetitle
\end{frame}

\subsection{git cherry}
\begin{frame}[fragile]
    \subslidetitle
\end{frame}

\subsection{Exercises}
\begin{frame}[fragile]
  \subslidetitle
\end{frame}
