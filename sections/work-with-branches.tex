\section{Work with branches}
\begin{frame}[fragile]
    \slidetitle
\end{frame}

The \cmd{git branch} command list you all the local branches
\begin{lstlisting}
$ (*\textcolor[HTML]{0000AA}{git branch}*)
* (*\textcolor{green}{master}*)
\end{lstlisting}

To create a new \textbf{test} branch, append the branch name to the command.
\begin{lstlisting}
$ (*\textcolor[HTML]{0000AA}{git branch test}*)
$ (*\textcolor[HTML]{0000AA}{git branch}*)
* (*\textcolor{green}{master}*)
  test
\end{lstlisting}

Note: the * character indicate which branch we are currently working on.
\newline
To delete an existing branch use the -d or -D flag:
\begin{lstlisting}
$ (*\textcolor[HTML]{0000AA}{git branch test -d}*)
Deleted branch test (was 5c81bc0).
$ (*\textcolor[HTML]{0000AA}{git branch}*)
* (*\textcolor{green}{master}*)
\end{lstlisting}
\end{frame}

Note: you cannot delete the branch you are currently working on.

\subsection{git checkout}
\begin{frame}[fragile]
    \subslidetitle
% -b
\end{frame}

\subsection{git stash}
\begin{frame}[fragile]
    \subslidetitle
% clean, pop, 
\end{frame}

\subsection{git diff}
\begin{frame}[fragile]
    \subslidetitle
\end{frame}

\subsection{git rebase}
\begin{frame}[fragile]
    \subslidetitle
% -i
\end{frame}

\subsection{git merge}
\begin{frame}[fragile]
    \subslidetitle
\end{frame}

\subsection{git cherry}
\begin{frame}[fragile]
    \subslidetitle
\end{frame}

\subsection{Exercises}
\begin{frame}[fragile]
  \subslidetitle
\end{frame}
