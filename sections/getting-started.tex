\section{Getting started}
\begin{frame}[fragile]
  \slidetitle

  In this section, we will learn how to:
  \begin{itemize}
    \item Configure git
    \item Clone a git repository
    \item Add a file to a repository
    \item Commit changes
  \end{itemize}
    %A git repository contains a folder called {\bf .git}. There are two option to get a git repository:
%\begin{itemize}
%\item create a new repository using \cmd{git init}
%\item clone an existing git repository using \cmd{git clone}
%\end{itemize}
\end{frame}

\subsection{Configuring git}
\begin{frame}[fragile]
  \subslidetitle
  The global git configuration is stored in \cmd{~/.gitconfig}.
  \\
  \vspace{1em}

  The command \cmd{git config} can used to configure git:
  \begin{lstlisting}
$ (*\textcolor[HTML]{0000AA}{git config --global add user.name "My name"}*)
$ (*\textcolor[HTML]{0000AA}{git config --global add user.email "myemail@git.ch"}*)
  \end{lstlisting}

  Other git settings:
  \begin{lstlisting}[basicstyle=\small\ttfamily\bfseries]
$ (*\textcolor[HTML]{0000AA}{cat ~/.gitconfig}*)
[color]
  diff = auto
  status = auto
  branch = auto
  grep = auto

[alias]
  st = status
  co = checkout
  ci = commit
  br = branch
  l = log --graph --pretty=format:'%C(yellow)%h%C(cyan)%d%Creset %s %C(white)- %an, %ar%Creset'
  ll = log --stat --abbrev-commit

[merge]
  tool = kdiff3
  \end{lstlisting}

\end{frame}

%\subsection{git init}
%\begin{frame}[fragile]
  %\subslidetitle
  %The command \cmd{git init} is used to create a new repository.
  %The following steps create a new repository:
  %\begin{itemize}
  %\item run \cmd{git init} inside your project
    %\begin{itemize}
    %\item creates .git directory
    %\end{itemize}
  %\item Add files to git repository using \cmd{git add .}
  %\item Optional: add .gitignore file
  %\item Run {git commit} for initial commit
    %\begin{itemize}
    %\item creates branch called 'master'
    %\end{itemize}
  %\end{itemize}

  %\begin{lstlisting}
%cd myproject
%git init
%git add .
%git commit
  %\end{lstlisting}
%\end{frame}

\subsection{Clone a git repository}
\begin{frame}[fragile]
  \subslidetitle
  The command \cmd{git clone} is used to download a whole git repository.
  \\
  \vspace{1em}
  Clone the gitmoon repository:
  \begin{lstlisting}
$ (*\textcolor[HTML]{0000AA}{git clone https://github.com/neolynx/gitmoon.git}*)
  \end{lstlisting}

% protocols
%git protocols:
%\begin{itemize}
%\item http[s]
%\item ssh
%\item git
%\item ftp rsync ...
%\end{itemize}

  Let's see what we find ...
  \begin{lstlisting}
$ (*\textcolor[HTML]{0000AA}{cd gitmoon}*)
$ (*\textcolor[HTML]{0000AA}{ls}*)
moon_1024.jpg  moon.html  moon.js  three.min.js
  \end{lstlisting}

  Oh, there is a HTML file !
  \begin{lstlisting}
$ (*\textcolor[HTML]{0000AA}{firefox moon.html \&}*)
  \end{lstlisting}

\end{frame}

\subsection{Show the git status}
\begin{frame}[fragile]
  \subslidetitle
  \begin{lstlisting}
$ (*\textcolor[HTML]{0000AA}{git status}*)
On branch master
Your branch is up-to-date with 'origin/master'.
nothing to commit, working directory clean
  \end{lstlisting}
\end{frame}


\subsection{Add a file to the repository}
\begin{frame}[fragile]
  \subslidetitle

  Let's create an AUTHORS file with your name:
  \begin{lstlisting}
$ (*\textcolor[HTML]{0000AA}{echo Tux Penguin > AUTHORS}*)
  \end{lstlisting}

  Check the git status:
  \begin{lstlisting}
$ (*\textcolor[HTML]{0000AA}{git status}*)
On branch master
Your branch is up-to-date with 'origin/master'.
Untracked files:
  (use "git add <file>..." to include in what will be committed)

        (*\textcolor{red}{AUTHORS}*)

nothing added to commit but untracked files present (use "git add" to track)
  \end{lstlisting}

\end{frame}
% --branch (different from master)
% --depth (shallow clone)
% --recursive (submodules)

\subsection{git add}
\begin{frame}[fragile]
  \subslidetitle

  The command \cmd{git add} tells git to track the file:
  \begin{lstlisting}
$ (*\textcolor[HTML]{0000AA}{git add AUTHORS}*)
  \end{lstlisting}

  Check the git status:
  \begin{lstlisting}
$ (*\textcolor[HTML]{0000AA}{git status}*)
On branch master
Your branch is up-to-date with 'origin/master'.
Changes to be committed:
  (use "git reset HEAD <file>..." to unstage)

        (*\textcolor[HTML]{00AA00}{new file:   AUTHORS}*)
  \end{lstlisting}

\end{frame}


\subsection{git commit}
\begin{frame}[fragile]
  \subslidetitle

  The command \cmd{git commit} tells git to commit the file:
  \begin{lstlisting}
$ (*\textcolor[HTML]{0000AA}{git commit -m 'add authors file'}*)

[master c01af7f] add authors file
 1 file changed, 1 insertion(+)
 create mode 100644 AUTHORS
\end{lstlisting}

  Check the git status:
  \begin{lstlisting}
$ (*\textcolor[HTML]{0000AA}{git status}*)
On branch master
Your branch is ahead of 'origin/master' by 1 commit.
  (use "git push" to publish your local commits)
nothing to commit, working directory clean
  \end{lstlisting}

\end{frame}

