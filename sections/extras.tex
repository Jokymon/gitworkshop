\section{Extras}
\begin{frame}[fragile]
  \slidetitle
  This section covers some advanced topics:
  \begin{itemize}
    \item Cleaning workspace
    \item Working with patches
    \item Search keywords
    \item ...
  \end{itemize}
\end{frame}

\subsection{Remove untracked files}
\begin{frame}[fragile]
    \subslidetitle
  Tracking a project under git helps you to keep your workspace clean, after your compilation process generated some temporary files:

  \begin{lstlisting}
(*\textcolor[HTML]{18B2B2}{(master)}*) $ (*\textcolor[HTML]{0000AA}{touch \$(echo "compile" | hexdump | head -n1)}*)
(*\textcolor[HTML]{18B2B2}{(master)}*) $ (*\textcolor[HTML]{0000AA}{git status}*)
On branch master
...
(*\textcolor[HTML]{AA0000}{	0000000}*)
(*\textcolor[HTML]{AA0000}{	0a65}*)
(*\textcolor[HTML]{AA0000}{	6c69}*)
(*\textcolor[HTML]{AA0000}{	6f63}*)
(*\textcolor[HTML]{AA0000}{	706d}*)
...
\end{lstlisting}
  Now we would like to clean all this temporary files, using \cmd{git clean}:
  \begin{lstlisting}
(*\textcolor[HTML]{18B2B2}{(master)}*) $ (*\textcolor[HTML]{0000AA}{git git clean -df}*)
Removing 0000000
Removing 0a65
...
\end{lstlisting}

\end{frame}

\subsection{Search with git}
\begin{frame}[fragile]
  \subslidetitle
  Lost in search? It is just convenient to have \cmd{git grep}:
  \begin{lstlisting}
(*\textcolor[HTML]{18B2B2}{(master)}*) $ (*\textcolor[HTML]{0000AA}{git grep moon.js}*)
moon.html:        <script src="(*\textcolor[HTML]{AA0000}{moon.js}*)"></script>
\end{lstlisting}

  Or we can search commits that touch lines containing the keyword you are looking for using \cmd{git log --pickaxe-regexp}:
  \begin{lstlisting}
(*\textcolor[HTML]{18B2B2}{(master)}*) $ (*\textcolor[HTML]{0000AA}{git log --oneline --abbrev --pickaxe-regex "*moon*"}*)
(*\textcolor[HTML]{ae6617}{a3c399f}*) remove the blue moon
(*\textcolor[HTML]{ae6617}{93ea12c}*) change the green moon color to red
...
39719c9 initial commit
\end{lstlisting}

  Finally a option of \cmd{git branch} to search the branch a commit belongs to:
  \begin{lstlisting}
(*\textcolor[HTML]{18B2B2}{(master)}*) $ (*\textcolor[HTML]{0000AA}{git branch --contains a3c399f}*)
(*\textcolor[HTML]{00AA00}{* master}*)
\end{lstlisting}
\end{frame}

\subsection{Finger pointing!?}
\begin{frame}[fragile]
  \subslidetitle

  The \cmd{git blame} returns revision and author information per each line of a given file:
  \begin{lstlisting}
(*\textcolor[HTML]{18B2B2}{(master)}*) $ (*\textcolor[HTML]{0000AA}{git blame moon.js}*)
...
^412af06 (Andre Roth  2015-11-28 14:41:45 +0100  2) var moons = [];
^412af06 (Andre Roth  2015-11-28 14:41:45 +0100  3)
^412af06 (Andre Roth  2015-11-28 14:41:45 +0100  4) init();
ff9e4cd5 (Eric Keller 2015-12-13 18:53:56 +0100  5) moon( "blue" );
429131d4 (Eric Keller 2015-12-13 22:06:06 +0100  6) moon( "black" );
bf208729 (Eric Keller 2015-12-13 22:01:30 +0100  7) moon( "red" );
^412af06 (Andre Roth  2015-11-28 14:41:45 +0100  8) animate();
...
\end{lstlisting}

\end{frame}

\subsection{Work with patches}
\begin{frame}[fragile]
    \subslidetitle

  Remove the blue moon:
  \begin{lstlisting}
(*\textcolor[HTML]{18B2B2}{(master)}*) $ (*\textcolor[HTML]{0000AA}{sed -i "/blue/d" moon.js}*)
(*\textcolor[HTML]{18B2B2}{(master)}*) $ (*\textcolor[HTML]{0000AA}{git commit -a -m "remove the blue moon"}*)
\end{lstlisting}

  The \cmd{git format-patch} generates patch file to be send to 3rd party collaborator, or mailing list:
  \begin{lstlisting}
(*\textcolor[HTML]{18B2B2}{(master)}*) $ (*\textcolor[HTML]{0000AA}{git format-patch -1}*)
0001-remove-the-blue-moon.patch
\end{lstlisting}

  \begin{lstlisting}
(*\textcolor[HTML]{18B2B2}{(master)}*) $ (*\textcolor[HTML]{0000AA}{cat 0001-remove-the-blue-moon.patch}*)
From 7afa035a39c2dc9648b182772eee06e3181ae24e Mon Sep 17 00:00:00 2001
From: Eric Keller <keller.eric@gmail.com>
Date: Sun, 29 Nov 2015 10:06:27 +0000
Subject: [PATCH] remove the blue moon
...
 init();
-moon( "blue" );
 moon( "white" );
...
\end{lstlisting}

\end{frame}

\subsection{Apply a patch}
\begin{frame}[fragile]
    \subslidetitle
  First remove the last commit:
  \begin{lstlisting}
(*\textcolor[HTML]{18B2B2}{(master)}*) $ (*\textcolor[HTML]{0000AA}{git reset --hard HEAD\textasciicircum1}*)
\end{lstlisting}
  Create new no-blue branch in order to apply this patch
  \begin{lstlisting}
(*\textcolor[HTML]{18B2B2}{(master)}*) $ (*\textcolor[HTML]{0000AA}{git checkout -b no-blue}*)
Switched to a new branch 'no-blue'
\end{lstlisting}

  The \cmd{git am} command takes a list of patch to apply to the current working area:
  \begin{lstlisting}
(*\textcolor[HTML]{18B2B2}{(no-blue)}*) $ (*\textcolor[HTML]{0000AA}{git am -3 0001-remove-the-blue-moon.patch}*)
Applying: remove the blue moon
\end{lstlisting}

  Note: As well as when merging branch the \cmd{git am} could potentially end up with conflict, therefore we would like to enforce a 3way-merge with the \cmd{-3} option.

\end{frame}

\subsection{Cherrypick a commit}
\begin{frame}[fragile]
    \subslidetitle
  You are currently implementing a bugfix on a released version and should simultaneously fix it on top of your development, seems to be a lot of work! Not if you use \cmd{git cherry-pick}:
  \begin{lstlisting}
(*\textcolor[HTML]{18B2B2}{(no-blue)}*) $ (*\textcolor[HTML]{0000AA}{git log -1 --oneline}*)
(*\textcolor[HTML]{ae6617}{1d36adc}*) remove the blue moon
\end{lstlisting}
  Change to the master branch:
  \begin{lstlisting}
(*\textcolor[HTML]{18B2B2}{(no-blue)}*) $ (*\textcolor[HTML]{0000AA}{git checkout master}*)
Switched to branch 'master'
\end{lstlisting}
  Now we try to cherry-pick the preceding commit:
  \begin{lstlisting}
(*\textcolor[HTML]{18B2B2}{(master)}*) $ (*\textcolor[HTML]{0000AA}{git cherry-pick 1d36adc}*)
[master fbd4ca8] remove the blue moon
 Date: Sun Nov 29 10:06:27 2015 +0000
 1 file changed, 1 deletion(-)
\end{lstlisting}
  Note:  The \cmd{git cherry-pick} could potencially end up with conflict.
\end{frame}

\subsection{Advanced undo}
\begin{frame}[fragile]
    \subslidetitle

  Want to undo a rebase operation? You deleted an important file? Got interrupted in the middle of an octopus merge... Do not worry \cmd{git reflog} is here to help.

  \begin{lstlisting}
(*\textcolor[HTML]{18B2B2}{(master)}*) $ (*\textcolor[HTML]{0000AA}{git reflog}*)
(*\textcolor[HTML]{ae6617}{9ed88d3}*) HEAD@{0}: checkout: moving from death-star to master
(*\textcolor[HTML]{ae6617}{429131d}*) HEAD@{1}: rebase finished: returning to refs/heads/death-star
(*\textcolor[HTML]{ae6617}{429131d}*) HEAD@{2}: rebase: add death-star
(*\textcolor[HTML]{ae6617}{9ed88d3}*) HEAD@{3}: rebase: checkout master
(*\textcolor[HTML]{ae6617}{9dd4b4b}*) HEAD@{4}: checkout: moving from master to death-star
(*\textcolor[HTML]{ae6617}{9ed88d3}*) HEAD@{5}: commit: snow white comment
\end{lstlisting}

  You use reflog in combination with the \cmd{git reset --hard HEAD@\{4\}} command, like a way back time machine.
\end{frame}

\subsection{Git submodules}
\begin{frame}[fragile]
  \subslidetitle
  Git offers the possibility to include other git repositories in a subdirectory of a project.
  \\
  \vspace{1em}
  The following commands deal with submodules:
  \begin{itemize}
    \item \cmd{git clone URL --recursive}
    \item \cmd{git submodule add URL PATH}
    \item \cmd{git submodule update}
    \item \cmd{git submodule sync}
  \end{itemize}
  \vspace{1em}
  Note: the main project can commit the subdirectory and fix the commit of the submodule.
\end{frame}

\subsection{Git aliases}
\begin{frame}[fragile]
  \subslidetitle

  Git allows to create aliases for commands.
  \\
  \vspace{1em}
  Example:
  instead of typing \cmd{git status} we want to type \cmd{git st}.
  \\
  \vspace{1em}
  Define aliases:

  \begin{lstlisting}
  git config --global alias.st = status
  git config --global alias.co = checkout
  git config --global alias.ci = commit
  git config --global alias.br = branch
  \end{lstlisting}

\end{frame}

\subsection{Example $\sim$/.gitconfig}
\begin{frame}[fragile]
  \subslidetitle

\begin{lstlisting}
[user]
  name = Tux Penguin
  email = tux@penguin

[color]
  diff = auto
  status = auto
  branch = auto
  grep = auto

[alias]
  st = status
  co = checkout
  ci = commit
  br = branch
  l = log --graph --pretty=format:'%C(yellow)%h%C(cyan)%d%Creset %s %C(white)- %an, %ar%Creset'
  ll = log --stat --abbrev-commit
\end{lstlisting}
\end{frame}
