\section{Git remotes}
\begin{frame}[fragile]
  \slidetitle
  This section covers the following topics:
  \begin{itemize}
    \pause
    \item Git remote repository concept
    \pause
    \item Add remote repositories
    \pause
    \item Synchronize with remote repositories
  \end{itemize}
\end{frame}

\subsection{Remote repositories}
\begin{frame}[fragile]
  \subslidetitle
  The \cmd{git remote} allows to manage remote repositories:
  \begin{lstlisting}
$ (*\textcolor[HTML]{0000AA}{git remote show}*)
origin
$ (*\textcolor[HTML]{0000AA}{git remote show origin}*)
* remote origin
  Fetch URL: https://github.com/neolynx/gitmoon.git
  Push  URL: https://github.com/neolynx/gitmoon.git
  HEAD branch: master
  Remote branch:
    master tracked
  Local branch configured for 'git pull':
    master merges with remote master
  Local ref configured for 'git push':
    master pushes to master (up to date)
\end{lstlisting}
  \vspace{1em}
  Note: the default remote is called \textbf{origin}
\end{frame}

\subsection{Git protocols}
\begin{frame}[fragile]
  \subslidetitle
  Git repositories can be accessed locally or over the network.
  \\
  \vspace{1em}
  Various protocols are supported:
  \begin{itemize}
  \opt{local}  {file system based}
  \opt{http[s]}{good for read only access without password}
  \opt{ssh}    {normally used for read-write access}
  \opt{git}    {git native protocol on port 9418}
  \opt{legacy} {ftp, rsync, ...}
  \end{itemize}
\end{frame}

\subsection{Configure SSH access}
\begin{frame}[fragile]
  \subslidetitle
  Create a SSH key pair:
  \begin{lstlisting}
$ (*\textcolor[HTML]{0000AA}{ssh-keygen}*)
Generating public/private rsa key pair.
Enter file in which to save the key (...): (*\textcolor[HTML]{0000AA}{<enter>}*)
Created directory '.../.ssh'.
Enter passphrase (empty for no passphrase): (*\textcolor[HTML]{0000AA}{<enter>}*)
Enter same passphrase again: (*\textcolor[HTML]{0000AA}{<enter>}*)
Your identification has been saved in .../.ssh/id_rsa.
Your public key has been saved in .../.ssh/id_rsa.pub.
The key fingerprint is:
7e:f8:15:2a:b3:a2:9c:30:4e:c7:60:50:a4:d5:a9:82 user@host
The key's randomart image is:
+--[ RSA 2048]----+
|   . . .         |
|  . = = S        |
|   = X * X O o   |
...
\end{lstlisting}
\end{frame}

\subsection{Configure SSH access}
\begin{frame}[fragile]
  \subslidetitle
  Display your public key:
  \begin{lstlisting}
$ (*\textcolor[HTML]{0000AA}{cat \textasciitilde/.ssh/id\_rsa.pub}*)
ssh-rsa AAAAB3NzaC1yc2EAAAADAQABAAABAQDOmt7Y4H51gc2m
GmZsFzES6shVLFLEJ/lFCTwyosWHYDaluK71nGCelp61oTocgf4N
HBwTZmo0EZ1k0RHYt8Q3LF8e5fbC+dXt5E35XtkVFuUC7IG2/6fm
NW41j3lw9UUVrOBDgx+QvvoCuRQaxNd4mRaLsRbj9WXt17hGuNNW
ioKPWLSpw/4KHJ34hCrnliAQJ+jlW/0ieOooFp057diCka6Jn7BW
jXHi8sWMxIfyPyV2+4Kt8OpChFNYjzaL5LMRRhMnvJ8zP5SFJB2q
HP50zPYQ+gKoSda7GZedZRgD7gT7ir/u8X9HSpNyTNTafhp9+3Aj
uUiYLTgtczTgYk/T user@host
\end{lstlisting}

  This whole output can be added to the SSH access keys
  section in the web front end of your git appliance.
\end{frame}

\subsection{Exercises}
\begin{frame}[fragile]
  \subslidetitle
  Create a commit for each exercise below:
  \begin{exercise}
    \item Create a SSH key pair
    \item Configure SSH access in your git appliance
    \item Create a new private repository, called \textbf{gitmoon}
    \item Clone the private repository to a directory \textbf{mygitmoon} \\
          \vspace{1em}
          Hint: switch to your home directory first, by typing \cmd{cd}
    \item Go back to the original gitmoon project (\cmd{cd \textasciitilde/gitmoon})
  \end{exercise}
\end{frame}

\subsection{Adding a remote repository}
\begin{frame}[fragile]
  \subslidetitle
  Let's add the private repository as remote \textbf{upstream}:
  \begin{lstlisting}
$ (*\textcolor[HTML]{0000AA}{git remote add upstream} \textcolor[HTML]{444444}{URL}*)
$ (*\textcolor[HTML]{0000AA}{git remote show}*)
origin
upstream
\end{lstlisting}
\end{frame}

\subsection{Pushing commits to a remote}
\begin{frame}[fragile]
  \subslidetitle
  Now, we can push our local commits to the \textbf{upstream} remote:
  \begin{lstlisting}
$ (*\textcolor[HTML]{0000AA}{git push upstream master}*)
git push upstream master
Counting objects: 6, done.
Delta compression using up to 2 threads.
Compressing objects: 100% (6/6), done.
Writing objects: 100% (6/6), 331.07 KiB | 0 bytes/s, done.
Total 6 (delta 0), reused 0 (delta 0)
remote: ...
To (*\textcolor[HTML]{444444}{URL}*)
 * [new branch]      master -> master
\end{lstlisting}
\end{frame}

\subsection{Pulling commits from a remote}
\begin{frame}[fragile]
  \subslidetitle
  In our \textbf{mygitmoon} project, these commits can now be pulled:
  \begin{lstlisting}
$ (*\textcolor[HTML]{0000AA}{cd \textasciitilde/mygitmoon}*)
$ (*\textcolor[HTML]{0000AA}{git pull}*)
remote: Counting objects: 6, done.
remote: Compressing objects: 100% (6/6), done.
remote: Total 6 (delta 0), reused 0 (delta 0)
Unpacking objects: 100% (6/6), done.
From (*\textcolor[HTML]{444444}{URL}*)
 * [new branch]      master     -> origin/master
\end{lstlisting}
\end{frame}

\subsection{git fetch}
\begin{frame}[fragile]
  \subslidetitle
\end{frame}

\subsection{Stashing modifications}
\begin{frame}[fragile]
  \subslidetitle

  Imagine you are in the middle of a modification.

  \begin{lstlisting}
$ (*\textcolor[HTML]{0000AA}{git status}*)
...
       (*\textcolor[HTML]{AA0000}{modified:   AUTHORS}*)
...
\end{lstlisting}

  And suddenly something really important needs to be done immediately!

  \begin{lstlisting}
$ (*\textcolor[HTML]{0000AA}{git stash}*)
Saved working directory and index state WIP on master: a7281d1 add authors file
HEAD is now at a7281d1 add authors file
$ (*\textcolor[HTML]{0000AA}{git status}*)
# On branch master
nothing to commit (working directory clean)
\end{lstlisting}
  Note: you now have a clean working directory again.
\end{frame}


\subsection{Restore changes}
\begin{frame}[fragile]
  \subslidetitle

  Do not worry, \textbf{git stash} does not loose any information, let's have a look:

  \begin{lstlisting}
$ (*\textcolor[HTML]{0000AA}{git stash list}*)
stash@{0}: WIP on master: a7281d1 add authors file
(END)
\end{lstlisting}

  Restore the changes from the stash queue:

  \begin{lstlisting}
$ (*\textcolor[HTML]{0000AA}{git stash pop}*)
On branch master
Changes not staged for commit:
   (use "git add <file>..." to update what will be committed)
   (use "git checkout -- <file>..." to discard changes in working directory)

       (*\textcolor[HTML]{AA0000}{modified:   AUTHORS}*)

no changes added to commit (use "git add" and/or "git commit -a")
Dropped refs/stash@{0} (1638e645e67ff57ffc76b2c79f851ec894ddd74c)
\end{lstlisting}
\end{frame}

\subsection{git origin}
\begin{frame}[fragile]
  \subslidetitle
  ??
\end{frame}

\subsection{git remote update}
\begin{frame}[fragile]
  \subslidetitle
  ??
\end{frame}

