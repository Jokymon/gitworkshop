\section{Git repository}
\begin{frame}[fragile]
    \slidetitle
    A git repository contains a folder called {\bf .git}. There are two option to get a git repository:
\begin{itemize}
\item create a new repository using \cmd{git init}
\item clone an existing git repository using \cmd{git clone}
\end{itemize}
\end{frame}

\subsection{git init}
\begin{frame}[fragile]
  \subslidetitle
  The command \cmd{git init} is used to create a new repository.
  The following steps create a new repository:
  \begin{itemize}
  \item run \cmd{git init} inside your project
    \begin{itemize}
    \item creates .git directory
    \end{itemize}
  \item Add files to git repository using \cmd{git add .}
  \item Optional: add .gitignore file
  \item Run {git commit} for initial commit
    \begin{itemize}
    \item creates branch called 'master'
    \end{itemize}
  \end{itemize}

  \begin{lstlisting}
cd myproject
git init
git add .
git commit
  \end{lstlisting}
\end{frame}

\subsection{git clone}
\begin{frame}[fragile]
  \subslidetitle
  The command \cmd{git clone [option(s)] <repository> [<directory>]} is used to download a repository's whole history.
  \begin{lstlisting}
git clone https://github.com/git/git.git
  \end{lstlisting}

% protocols
git protocols:
\begin{itemize}
\item http[s]
\item ssh
\item git
\item ftp rsync ...
\end{itemize}
\end{frame}

% --local (transport mechanisms)
\subsection{git clone --branch}
\begin{frame}[fragile]
  \subslidetitle
\end{frame}
% --branch (different from master)
% --depth (shallow clone)
% --recursive (submodules)

\subsection{Exercises}
\begin{frame}[fragile]
  \subslidetitle
\end{frame}
