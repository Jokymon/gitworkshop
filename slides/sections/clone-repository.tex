\section{Git repository}
\begin{frame}[fragile]
    \slidetitle
    A git repository contains a folder called {\bf .git}. There are two option to get a git repository:
\begin{itemize}
\item create a new repository using \cmd{git init}
\item clone an existing git repository using \cmd{git clone}
\end{itemize}
\end{frame}

\subsection{git clone}
\begin{frame}[fragile]
  \subslidetitle
  The command \cmd{git clone [option(s)] <repository> [<directory>]} is used to download a repository's whole history.
  \begin{lstlisting}
git clone https://github.com/git/git.git
  \end{lstlisting}

% protocols
git protocols:
\begin{itemize}
\item http[s]
\item ssh
\item git
\item ftp rsync ...
\end{itemize}
\end{frame}

% --local (transport mechanisms)
\subsection{git clone --branch}
\begin{frame}[fragile]
  \subslidetitle
\end{frame}
% --branch (different from master)
% --depth (shallow clone)
% --recursive (submodules)

\subsection{Exercises}
\begin{frame}[fragile]
  \subslidetitle
\end{frame}
