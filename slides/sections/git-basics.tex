\section{Getting started with git}
\begin{frame}[fragile]
  \slidetitle

  Learn how to:
  \begin{itemize}
    \item Configure git
    \item Clone a git repository
    \item Add a file to a repository
    \item Commit changes
  \end{itemize}
    %A git repository contains a folder called {\bf .git}. There are two option to get a git repository:
%\begin{itemize}
%\item create a new repository using \cmd{git init}
%\item clone an existing git repository using \cmd{git clone}
%\end{itemize}
\end{frame}

\subsection{Configure git}
\begin{frame}[fragile]
  \subslidetitle
  Git needs to know a name and an email address in order to create commits.
  \vspace{1em}
  Use the following commands to configure git:
  \begin{itemize}
      \option{git config --global add user.name "My name"}{}
      \option{git config --global add user.email "myemail@git.ch"}{}
  \end{itemize}

  The configuration is stored globally in \bf{$\sim$/.gitconfig}
\end{frame}

%\subsection{git init}
%\begin{frame}[fragile]
  %\subslidetitle
  %The command \cmd{git init} is used to create a new repository.
  %The following steps create a new repository:
  %\begin{itemize}
  %\item run \cmd{git init} inside your project
    %\begin{itemize}
    %\item creates .git directory
    %\end{itemize}
  %\item Add files to git repository using \cmd{git add .}
  %\item Optional: add .gitignore file
  %\item Run {git commit} for initial commit
    %\begin{itemize}
    %\item creates branch called 'master'
    %\end{itemize}
  %\end{itemize}

  %\begin{lstlisting}
%cd myproject
%git init
%git add .
%git commit
  %\end{lstlisting}
%\end{frame}

\subsection{Clone a git repository}
\begin{frame}[fragile]
  \subslidetitle
  The command \cmd{git clone [option(s)] <repository> [<directory>]} is used to download a repository's whole history.
  \begin{lstlisting}
git clone https://github.com/neolynx/gitmoon.git
  \end{lstlisting}

% protocols
%git protocols:
%\begin{itemize}
%\item http[s]
%\item ssh
%\item git
%\item ftp rsync ...
%\end{itemize}

  Let's see what we find ...
  \begin{lstlisting}
$ cd gitmoon
$ ls
moon_1024.jpg  moon.html  moon.js  three.min.js
  \end{lstlisting}

  Oh, there is a HTML file !
  \begin{lstlisting}
$ firefox moon.html &
  \end{lstlisting}

\end{frame}

\subsection{Show the git status}
\begin{frame}[fragile]
  \subslidetitle
  \begin{lstlisting}
$ git status
On branch master
Your branch is up-to-date with 'origin/master'.
nothing to commit, working directory clean
  \end{lstlisting}
\end{frame}


\subsection{Add a file to the repository}
\begin{frame}[fragile]
  \subslidetitle

  Let's create an AUTHORS file with the username:
  \begin{lstlisting}
$ echo $USER > AUTHORS
  \end{lstlisting}

  Check the git status:
  \begin{lstlisting}
$ git st
On branch master
Your branch is up-to-date with 'origin/master'.
Untracked files:
  (use "git add <file>..." to include in what will be committed)

        AUTHORS

nothing added to commit but untracked files present (use "git add" to track)
  \end{lstlisting}

\end{frame}
% --branch (different from master)
% --depth (shallow clone)
% --recursive (submodules)





\subsection{Exercises}
\begin{frame}[fragile]
  \subslidetitle
\end{frame}

\subsection{git bash prompt}
\begin{frame}[fragile]
  \subslidetitle
\end{frame}
