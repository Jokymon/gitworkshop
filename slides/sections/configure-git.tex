\section{Configure git}
\begin{frame}[fragile]
  \slidetitle
  \begin{block}{Getting started with git}
    Once you have git installed you are able to configure git on 3 different levels:
    \begin{itemize}
      \option{local repository level}{path/to/repository/.git/config}
      \option{user level}{ /home/\$USER/.gitconfig }
      \option{system level}{on GNU/Linux: /etc/gitconfig}
    \end{itemize}

    The configuration file is a simple text file and can be edited with any text editor.
    Priority of git config: local, user, system.
  \end{block}
\end{frame}

\subsection{git config}
\begin{frame}[fragile]
  \subslidetitle
  The command \cmd{git config} can be used to change the global, system and local config of git.
  \begin{itemize}
      \option{git config ..}{}
      \option{git config --global ..}{}
      \option{git config --system ..}{}
  \end{itemize}
\end{frame}

\subsection{Set username and email}
\begin{frame}[fragile]
  \subslidetitle
  \vspace{1em}
  Set your name and email locally:
  \begin{itemize}
      \option{git config add user.name "My name"}{}
      \option{git config add user.email "myemail@git.ch"}{}
  \end{itemize}
  \vspace{1em}
  Or globally:
  \begin{itemize}
      \option{git config --global add user.name "My name"}{}
      \option{git config --global add user.email "myemail@git.ch"}{}
  \end{itemize}

  Extract from \bf{$\sim$/.gitconfig}
\begin{lstlisting}
[user]
  name = Andreas Schmid
  email = ikeark@gmail.com
\end{lstlisting}
\end{frame}

\subsection{Create aliases}
\begin{frame}[fragile]
  \subslidetitle

  Git allows to create aliases for commands.

  Example:

  instead of typing \bf{git status} we want to type \bf{git st}.

  Define alias:
  \begin{lstlisting}
  git config --global alias.st = status
  git config --global alias.co = checkout
  git config --global alias.ci = commit
  git config --global alias.br = branch
  \end{lstlisting}


  Extract from \bf{$\sim$/.gitconfig}
\begin{lstlisting}
[alias]
  st = status
  co = checkout
  ci = commit
  br = branch
  l = log --graph --pretty=format:'%C(yellow)%h%C(cyan)%d%Creset %s %C(white)- %an, %ar%Creset'
  ll = log --stat --abbrev-commit
\end{lstlisting}
\end{frame}

\subsection{Set mergetool}
\begin{frame}[fragile]
  \subslidetitle
\begin{lstlisting}
[merge]
  tool = kdiff3
\end{lstlisting}
\end{frame}

\subsection{Example $\sim$/.gitconfig}
\begin{frame}[fragile]
  \subslidetitle
  The command \cmd{chgrp} is used to change the group:

  %\vspace{1em}
  %\cmd{chgrp \braces{OPTION}... GROUP FILE...}
  %\begin{itemize}
      %\option{-R}{apply permissions to every recursively to a directory}
  %\end{itemize}

\begin{lstlisting}
[user]
  name = Andreas Schmid
  email = ikeark@gmail.com

[core]
  pager = less -r

[color]
  diff = auto
  status = auto
  branch = auto
  grep = auto

[alias]
  st = status
  co = checkout
  ci = commit
  br = branch
  l = log --graph --pretty=format:'%C(yellow)%h%C(cyan)%d%Creset %s %C(white)- %an, %ar%Creset'
  ll = log --stat --abbrev-commit

[svn]
  addAuthorFrom = true
  useLogAuthor = true
  rmdir = true

[merge]
  tool = kdiff3

[http]
  sslVerify = false
\end{lstlisting}
\end{frame}

\subsection{Exercises}
\begin{frame}[fragile]
  \subslidetitle
    \begin{exercise}
    \item Configure git to use your name and email.
    \item Add alias \cmd{git co} for \cmd{git checkout}
    \end{exercise}
\end{frame}
